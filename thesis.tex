\documentclass[a4paper]{report}

%uncomment to see the references
%\usepackage{showkeys}
\usepackage[T2A]{fontenc}

\usepackage[section]{algorithm}
\usepackage{algorithmic}
\usepackage[english,russian]{babel}

\usepackage[backend=biber,bibencoding=utf8,sorting=none,sortcites=true,bibstyle=sty/gost71,maxnames=99,citestyle=numeric-comp,babel=other]{biblatex
}

\defbibenvironment{bibliography}
  {\list
     {\printfield[labelnumberwidth]{labelnumber}.}
     {\setlength{\labelwidth}{2\labelnumberwidth}%
      \setlength{\leftmargin}{\labelwidth}%
      \setlength{\labelsep}{\biblabelsep}%
      \addtolength{\leftmargin}{\labelsep}%
      \setlength{\itemsep}{\bibitemsep}%
      \setlength{\parsep}{\bibparsep}}%
      \renewcommand*{\makelabel}[1]{\hss##1}}
  {\endlist}
  {\itemm}



\usepackage[utf8]{inputenc}
\usepackage{csquotes}
%\usepackage{expdlist}
%\usepackage[nottoc,notbib]{tocbibind}
\usepackage[pdftex]{graphicx}
\graphicspath{{pic/}}
\usepackage{amsmath}
\usepackage{amssymb}
\usepackage{amsthm}
\usepackage{amsfonts}
\usepackage{amsxtra}
\usepackage{sty/dbl12}
%\usepackage{srcltx}
\usepackage{epsfig}
% \usepackage{verbatim}
\usepackage{sty/rac}
\usepackage{listings}
\usepackage{sty/scala}
\usepackage[singlelinecheck=false]{caption}

\usepackage{xcolor, colortbl}
\definecolor{light-gray}{RGB}{230,230,230}
\definecolor{dkgreen}{RGB}{0,154,0}	
\definecolor{gray}{RGB}{128,128,128}
\definecolor{mauve}{RGB}{149,0,210}
\definecolor{purpur}{RGB}{255,204,153}
%%%%%%%%%%%%%%%%%%%%%%%%%%%%%%%%%%%%%%%%%%%%%%%%%%%%%%%%%%%%%%%%%%%%%%%%%%%%%%

\captionsetup[figure]{justification=centering,   position=bottom, skip=0pt}
\captionsetup[table] {justification=raggedright, position=top,    skip=0pt}

% Redefine margins and other page formatting

\setlength{\oddsidemargin}{0.5in}

% Various theorem environments. All of the following have the same numbering
% system as theorem.

\theoremstyle{plain}
\newtheorem{theorem}{Теорема}
\newtheorem{prop}[theorem]{Утверждение}
\newtheorem{corollary}[theorem]{Следствие}
\newtheorem{lemma}[theorem]{Лемма}
\newtheorem{question}[theorem]{Вопрос}
\newtheorem{conjecture}[theorem]{Гипотеза}
\newtheorem{assumption}[theorem]{Предположение}

\theoremstyle{definition}
\newtheorem{definition}[theorem]{Определение}
\newtheorem{notation}[theorem]{Обозначение}
\newtheorem{condition}[theorem]{Условие}
\newtheorem{example}[theorem]{Пример}
%\newtheorem{algorithm}[theorem]{Алгоритм}
\floatname{algorithm}{Листинг}
\renewcommand{\algorithmicrequire}{\textbf{Вход:}}

%\newtheorem{introduction}[theorem]{Introduction}

\renewcommand{\proof}{\\\textbf{Доказательство.}~}
\renewcommand{\lstlistingname}{Листинг}
 
 
 
% Default settings for code listings
\lstset{frame=tb,
  language=Java,
  aboveskip=3mm,
  belowskip=3mm,
  showstringspaces=false,
  columns=flexible,
  basicstyle={\small\ttfamily},
  numbers=none,
  numberstyle=\tiny\color{gray},
  keywordstyle=\color{blue},
  commentstyle=\color{dkgreen},
  stringstyle=\color{mauve},
  escapeinside={\%*}{*)},
  frame=single,
  breaklines=true,
  breakatwhitespace=true
  tabsize=3,
  extendedchars=\true,
  inputencoding=utf8
}

\lstdefinestyle{XML} {
    language=XML,
    extendedchars=true, 
    breaklines=true,
    breakatwhitespace=true,
    emph={},
    emphstyle=\color{red},
    basicstyle={\small\ttfamily},
    columns=fullflexible,
    commentstyle=\color{gray}\upshape,
    morestring=[b]",
    morekeywords={xmlns,version,type,stop,start,status,xmlns:ns2},
}

\lstnewenvironment{snippet}[1][]%
{
   \noindent
   \minipage{\linewidth} 
   \vspace{0.5\baselineskip}
   \lstset{basicstyle=\ttfamily\footnotesize,frame=single,#1}}
{\endminipage}

%\theoremstyle{remark}
%\newtheorem{remark}[theorem]{Remark}
%\include{header}
%%%%%%%%%%%%%%%%%%%%%%%%%%%%%%%%%%%%%%%%%%%%%%%%%%%%%%%%%%%%%%%%%%%%%%%%%%%%%%%

\numberwithin{theorem}{chapter}        % Numbers theorems "x.y" where x
                                        % is the section number, y is the
                                        % theorem number

%\renewcommand{\thetheorem}{\arabic{chapter}.\arabic{theorem}}

%\makeatletter                          % This sequence of commands will
%\let\c@equation\c@theorem              % incorporate equation numbering
%\makeatother                           % into the theorem numbering scheme

%\renewcommand{\theenumi}{(\roman{enumi})}

%%%%%%%%%%%%%%%%%%%%%%%%%%%%%%%%%%%%%%%%%%%%%%%%%%%%%%%%%%%%%%%%%%%%%%%%%%%%%%

\binoppenalty=10000
\relpenalty=10000

\addbibresource{thesis.bib}

\begin{document}
\renewcommand{\thelstlisting}{\thesection.\arabic{lstlisting}}
% Begin the front matter as required by Rackham dissertation guidelines
\initializefrontsections

\pagestyle{title}

\begin{center}
Санкт-Петербургский национальный исследовательский университет \\ информационных технологий, механики и оптики

\vspace{2cm}

Факультет информационных технологий и программирования

Кафедра компьютерных технологий

\vspace{3cm}

{\Large Баев Дмитрий Олегович}

\vspace{2cm}

\vbox{\LARGE\bfseries
Разработка системы построения отчетов автотестов, написанных на разных языках программирования
}

\vspace{4cm}

{\Large Научный руководитель: Ерошенко Артем Михайлович, старший инженер по автоматизации тестирования, компания Яндекс}

\vspace{6cm}

Санкт-Петербург\\ 2014
\end{center}

\newpage

\setcounter{page}{3}
\pagestyle{plain}

\tableofcontents
%\listoffigures
% Chapters
\startthechapters
\startprefacepage

Тестирование программного обеспечения --- важная часть процесса разработки. Последнее время большинство проектов не могут обходиться без него. Тестирование --- это процесс исследования, испытания программного продукта, имеющий две различные цели:

\begin{itemize}
\item продемонстрировать разработчикам и заказчикам, что программа соответствует требованиям;
\item выявить ситуации, в которых поведение программы является неправильным, нежелательным или не соответствующим спецификации.  
\end{itemize}

Есть несколько уровней тестирования:

\begin{itemize}
\item Модульное тестирование (юнит-тестирование) --- тестируется минимально возможный для тестирования компонент, например, отдельный класс или функция. Часто модульное тестирование осуществляется разработчиками ПО.
\item Интеграционное тестирование --- тестируются интерфейсы между компонентами, подсистемами или системами. При наличии резерва времени на данной стадии тестирование ведётся итерационно, с постепенным подключением последующих подсистем.
\item Системное тестирование --- тестируется интегрированная система на ее соответствие требованиям.
\end{itemize}

По сути, процесс тестирования состоит из двух частей: непосредственно проведение тестов и анализ результатов. Если в случае модульного тестирование анализ результатов занимает незначительное время, то при интеграционном тестировании это весьма нетривиальная задача. И когда количество высокоуровневые тестов в проекте начинает расти, анализ результатов становится большой проблемой. Дело в том, что высокоуровневые тесты сильно отличаются от модульных, и обладают рядом особенностей:

\begin{itemize}
\item они затрагивают гораздо больше функциональности, что затрудняет локализацию проблемы; 
\item такие тесты воздействуют на систему через посредников, например, браузер;
\item таких тестов очень много, и зачастую приходится вводить дополнительную категоризацию. Это могут быть компоненты, области функциональности, критичность.
\end{itemize}

В рамках стандартной модели xUnit анализировать результаты таких тестов достаточно проблематично. Например, в ошибка фреймворк«Can not click on element «Search Button»» тесте на веб-интерфейс может произойти по следующим причинам:

\begin{itemize}
\item сервис не отвечает;
\item на странице нет элемента «Search Button»;
\item элемент «Search Button» есть, но не получается на него кликнуть.
\end{itemize}

А имея дополнительную информацию о ходе выполнения теста, например, лог работы сервиса и скриншот страницы, локализовать проблему гораздо легче. 

Отсюда возникает следующая задача: разработать такую систему, которая позволяет агрегировать дополнительную информацию о ходе выполнения тестов и строить отчет. 

В данной работе будет описан процесс разработки такой системы.

Во второй главе, на основании анализа различных систем построения отчетов автотестов, а также опыта написания тестирования, сформулированы основные принципы для организации системы.

В третьей главе приведена подробная архитектура Allure, позволяющая легко интегрироваться с любыми существующими тестовыми фреймворками и расширять имующийся функционал. Подробно описана интеграция новых фреймворков, и новых систем сборки.

В заключении дано описание текущего состояния разработки и перспективы ее развития.
\chapter{Постановка задачи}
\label{chapter1}

\section{Термины и понятия}

В данном разделе описаны термины, используемые других частях представленной работы. При этом смысл многих терминов сужен, по сравнению, с их обычным смыслом. Это связано, с тем, что данная работа ориентирована в первую очередь, разработку системы построения отчетов автотестов. В дальнейшем приведенные термины будут использоваться в указанных значениях, если не оговорено обратное.

\subsection{Тестирование}

{\bf Аттачмент (attachment)} ---
любая информация, например, скриншот или лог, которую надо сохранить вместе с результатами теста.  

{\bf История (user story, story)} ---
модуль, часть функциональности, из которых может состоять требование.

{\bf Контекст теста (test context, test fixture)} ---
все, что нужно тестируемой системе чтобы мы могли ее протестировать. Например, наглядно понятно, что такое контекст теста в тестовом фремворке RSpec:

\begin{itemize}
\item контекст --- множество фруктов содержащих = {яблоко, апельсин, грушу};
\item экспертиза --- удалим апельсин из множества фруктов;
\item проверка --- множество фруктов содержит = {яблоко, груша}.
\end{itemize}

{\bf Ошибка теста (test error)} ---
ошибка, возникающая в ходе выполнения теста. Например, ошибка может возникнуть в проверяемой системе, или в самом тесте. Также ошибка может возникнуть в окружении (например, в операционной системе, виртуальной машине). Как правило, ошибка в самом тесте, а не в проверяемой системе.

{\bf Падение теста (test failure)} ---
тест падает, когда в проверке утверждений актуальное значение не совпадает с ожидаемым. Обычно означает наличие ошибки в проверяемой системе.

{\bf Проблемно-ориентированное проектирование (DDD)} --- набор принципов и схем, помогающих разработчикам создавать изящные системы объектов. При правильном применении оно приводит к созданию программных абстракций, которые называются моделями предметных областей. В эти модели входит сложная бизнес-логика, устраняющая промежуток между реальными условиями области применения продукта и кодом.

{\bf Продуктовый тест} --- в данной работе автор под данным термином подразумевает высокоуровневые тесты, например, интеграционные и системные.

{\bf Разработка через тестирование (TDD, test-driven development)} ---
техника разработки программного обеспечения, которая основывается на повторении очень которких циклов:

\begin{itemize}
\item написание теста на новую/изменяемую функциональность;
\item имплементация функциональности. Тест должен пройти;
\item рефакторинг кода под соответствующие стандарты разработки.
\end{itemize}

{\bf Разработка через требования (BDD, behavior-driven development)} ---
Разновидность разработки через тестирование, сфокусированная на тестах в которых четко описаны ожидаемые требования к тестируемой системе. Упор делается на то, что тесты используются как документация работы системы.

{\bf Результат теста (test result)} ---
тест, или тест суит могут быт ьзапущены несколько раз, и каждый раз возращать различные результаты проверок.

{\bf Тест} ---
некоторая процедура, котороая может быть выполена вручную или автоматически, и может быть использована для проверки ожидаемых требований к тестируемой системы. Тест часто называют тесткейсом.

{\bf Тест кейс (test case)} ---
обычно синоним для понятия "тест". В xUnit это также может обозначать тестовый класс, как месtex boldто в которое содержит тестовые методы.

{\bf Тест прошел (test success)} ---
ситуация, в которой проверка каждого утверждения в тесте прошла успешна (актуальные значения совпали с ожидаемыми), и в процессе выполенения теста не произошло никаких ошибок теста.

{\bf Тест ран (test run)} ---
запуск некоторого числа тестов или тестсуитов. После выполнения тестов из тестрана, мы можем получить их результаты.

{\bf Тест суит (test suite)} ---
способ наименования некоторого числа тестов, которые могут быть запущены вместе.

{\bf Тестируемая cистема (System Under Test)} ---
любая вещь, которую мы проверяем, например, метод, класс, объект, приложение.

{\bf Требование (feature)} ---
часть функциональности развивающейся системы, которая может быть протестирована.

{\bf Шаг (step)} ---
некоторая логическая часть теста. Каждый тест может состоять из одного или нескольких шагов. Как правило, шаги отображают сценарий теста.

{\bf Шаг теста (test step)} ---
смотри "Шаг".

{\bf Экстремальное программирование (XP)} ---
одна из гибких методологий разработки программного обеспечения

{\bf xUnit} ---
под этим термином подразумевается любой член семейства инфраструктур автоматизации тестов (Test Automation Framework), применяемых для автоматизации созданных вручную сценариев тестов. Для большинства современных языков программирования существует как минимум одна реализация xUnit. Обычно для автоматизации применяется тот же язык, который использовался для написания тестируемой ситстемы. Хотя это не всегда так, использовать подобную стратегию проще, поскольку тесты легко получают доступ к программному интерфейсу тестируемой системы.

{\bf WebDriver} ---
утилита, позволяющая эмулировать действия пользователя в различных браузерах.

Большинство членов xUnit реализованы с использованием объектно-ориентированной парадигмы.

\subsection{Сокращения}

{\bf SUT} --- System Under Test, смотри "Тестируемая система".

\section{Проблематика}

В современном мире развитие идет очень быстро. Требования к продуктам часто меняются, и надо уметь успевать за этими изменениями. Для этого, в частности, важно сокращение длительности релизного цикла программ. И последнее время все чаще узким местом является тестирование. Для того, чтобы ускорить процесс тестирования, надо ускорить выполнение тестов, и сократить время анализа результатов тестирования. В данной работе рассматривается инструмент, который помогает решить вторую задачу - ускорение анализа результатов тестирования. Но обо всем по порядку.

\subsection{JUnit}

Последние 12 лет тесты пишутся с использованием фреймворков xUnit, в частности JUnit (в дальнейшем будет рассматриваться именно JUnit, как основа фремворков xUnit). JUnit предоставляет систему для запуска тестов, также предоставляет отчет для анализа результатов. 
Фремворк был разработан Кент Беком (Kent Beck), автором таких методологий разработки ПО как экстремальное программирование (XP) и разработка через тестирование (TDD), в 2002 году. 
Данный фремворк ориентирован прежде всего на написание модульных тестов, однако последнее время сильно увеличилось количество функциональных тестов. Это связано, прежде всего, с сильным развитием интерфейсов (в частности, web-интерфейсов). И в случае функциональных тестов данный фремворк предоставляет мало информации. Решением данной проблемы являлось появления методологии разработки через требования.

\subsection{Разработка через тестирование}

Методология разработки через тестирования комбинирует в себе основные техники и практики из TDD с идеями из DDD и объектно-ориентированным проектированием. В данном подходе основная задача ставится в описании требований (спецификаций) к тестируемой системе и дальнейшей проверки системы на удовлетворение этим требованиям. 

В скором времени начали появлятся фремворки, основанные на BDD. Как правило, в данных фремворках идет абстрагирование от кода тестов, и вынесение спецификаций к фремворку на уровень описания. Например, следующим образом выглыдит спецификация в JBehave:

\begin{lstlisting}
Given a 5 by 5 game
When I toggle the cell at (2, 3)
Then the grid should look like
.....
.....
.....
..X..
.....
When I toggle the cell at (2, 4)
Then the grid should look like
.....
.....
.....
..X..
..X..
When I toggle the cell at (2, 3)
Then the grid should look like
.....
.....
.....
.....
..X..
\end{lstlisting}

Именно идеи BDD были взяты в основу разработаного автором фремворка, получившего название Allure.

\subsection{Allure}

Первым, и самым важным отличием разработываемого фремворка было то, что он не выполняет тесты, а просто собирает информацию о ходе их выполнения. Также, разрабатываемый фремворк должен уметь предоставлять результаты как в виде BDD, так и в виде xUnit. Еще одной важной идеей было то, что отчет должен быть простым и понятным каждому. Это позволит ввести дополнительный уровень контроля над тестировщиками. В больших компаниях часто возникает проблема, когда тестировщик не в полной мере ответственно подходит к анализу результатов. А другому человеку будет сложно понять, что же конкретно тестируется, не разбираясь в коде тестов. 

Взяв за основу данные идеи было проведено исследование, которое позволило сформулировать более подробные требования к разрабатываемой системе.

\chapter{Исследование}
\label{chapter2}

В данной главе рассматривается теоритические аспекты разработки фремворка. Проводится исследование существующих систем, описываются требования к разрабатываемой системе.

\section{JUnit}

Рассмотрим более подробно тестовый фреймворк JUnit. Данный фреймворк впервые показал, как надо устраивать процесс тестирования (на самом деле, основные идеи были сформированы Кент Беком при разработке SUnit, но именно в лице JUnit эти идеи получили широкое распространение):

\begin{itemize}
\item тесты представляют из себя набор проверок утверждений;
\item тесты могут быть сгруппированы в суиты, для совместного запуска;
\item у каждого теста или суита может быть подготовка (set up) или завершение (tear down);
\item суиты объеденяются в тест ран.
\end{itemize}

Ниже можно посмотреть пример простейщего JUnit теста.

\begin{lstlisting}
public class SampleTest {

    @Test
    public void sampleTest() throws Exception {
        assertThat(4, is(2 + 2));
    }
}
\end{lstlisting}

Для семейства xUnit существует стандартный отчет Surefire. Есть несколько разных видов этого отчета, но суть у всех одна: отображается список всех тестов, у каждого теста есть статус, время выполнения и сообщение об об ошибке (в случае, если тест не прошел). 

Основным недостатком xUnit является атомарность теста, невозможность отобразить тестовый сценарий. Основаный на JUnit тестовый фремворк Thucydides решает эту проблему путем разбиения тестов на шаги.

\section{Thucydides}

Thucydides это фремворк для написание тестов на веб-интерфейс с использованием webDriver, написанный Джоном Смартом (John Ferguson Smart). Джон Смарт --- специалист в BDD, в оптимизации жизненного цикла процесса разработки. Хорошо известеный спикер множества интернациональных конференций, автор множества статей.

В свое время данный фреймворк произвел революцию. Прежде всего, фремворк предлагал структуру для тестов на веб-интерфейс, концепцию разбиения тестов на шаги и возможность сохранять скриншоты каждого шага. Шагом теста являлся любой метод, проаннотированный аннотацией @Step.
Фремворк анализировал структуру данных методов и отображал информацию о них в отчете. Мало того, это помогало сильно сократить код тестов --- шаги выносились в отдельные библиотеки и переиспользовались в множестве проектов.

Отчет, который строил Thucydides для тестов, могли посмотреть другие люди, и понять, что происходит в тесте. В Яндексе это позволило разделить тестировщиков на "автоматизаторов" и ответственных за релиз. Первые писали тесты, а вторые эти тесты запускали, просматривали результаты и, в случае необходимости, дополнительно проводили ручное тестирование продукта. Это позволило сильно увеличить качество тестирования за счет появления дополнительного уровня контроля тестировщиков.

Еще одной важной возможностью Thucydides являлось сохранение скриншотов. В тестах на веб-интерфейс очень важно иметь возможно увидеть, в чем проблема. Однако, не всегда хватает возможности сохранения только скриншотов. Хотелось бы уметь приклеплять к отчету и другие типы данных, например, логи.

Не обошлось в данном фреймворке без недостатков. Прежде всего, предложеная структура слишком ограничивала возможности тестирования. С использованием Thucydides можно писать тесты на веб-интерфейс, а это лишь малая часть функционального тестирования. Для тестов, например, на API данный фреймворк не подходит.

Также важным недостатком является ограничение в используемых технологиях: тесты можно писать только используя тестовый фреймворк JUnit и систему сборки Maven. Хотя, наверное, в любой большой компании тесты пишутся на разных языках программирования, в зависимости от специфики поставленной задачи.  

Кроме того, Thucydides предлагал очень много возможностей, со временем превращаясь в большой проект с множеством проблем. Правильным решением было бы сделать модульный проект, с возможностью подмены некоторых компонент.

\subsection{Итого} 

Thucydides имеет следующие плюсы: 
\begin{itemize}
\item задается единая структура тестов;
\item разбиение тестов на шаги, отображение сценариев тестов;
\item сохранение скриншотов;
\item хороший отчет, с возможностью группировки тестов по требованиям.
\end{itemize}

Но минусов тоже много:
\begin{itemize}
\item слишком много ограничений, заданных структурой тестов;
\item слишком много ограничений на структуру шагов;
\item невозможность использовать другие типы аттачментов;
\item большое количество проблем и ошибок;
\item отчет понятен только тестировщикам.
\end{itemize}

Вывод: отличный фремворк, если речь идет только о простом тестировании веб-интерфейсов с использованием JUnit и Maven. В иных случаях не подходит.

\section{Уточненные требования к работе}

Основываясь на анализе существующих решений, были выработаны уточненные требования к фремворку Allure:

\begin{itemize}
\item умение оперировать как в терминах xUnit, так и в терминах BDD;
\item отображение сценария теста, разбиение теста на шаги;
\item возможность приклеплять к результатам теста произвольные данные;
\item независимость от стека используемых технологий;
\item простой и понятный отчет, который смогут смотреть не только разработчики тестов.
\end{itemize}

В следующей главе рассказывается о процессе разработки фреймворка, удовлетворяющего данным требованиям, о проблемах, с которыми пришлось столкнуться автору данной работы в процессе разработки и описывается фреймворка на момент написания работы.
\chapter{Проектирование программного продукта} 
\label{chapter3}

В предыдущих главах были подробно описаны концепции положенные в основу фремворка Allure. В этой главе
описаны технические подробности реализации.

\section{Подключение к тестам}

Информацию, которую фремворк собирает о тестах, можно разделить на две группы:

\begin{itemize}
\item информация, которую можно получить, не меняя код тестов, например, имя теста, статус выполнения и время выполнения теста;
\item та информация, которую будет предоставлять тестировщик. Например, сценарий теста, описание теста и требования к тесту.
\end{itemize}

Большинство фремворков xUnit предоставляют интерфейс листенера, позволяющий собирать первый тип информации. Данный подход полностью удовлетворяет требованиям, так как подключение листенеров в большинстве случаев вынесено на уровень конфигурации запуска тестов.

Со вторым типом информации намного интереснее. В качестве примера рассмотрим реализацию для jUnit.

\subsection{Шаги}

В языке программирования Java шаг теста это любой метод, аннотированный аннотацией @Step. Для того, чтобы собрать информацию о пройденных шагах, надо выполнять некоторый код до и после каждого вызова такого метода. Автор воспользовался фреймворком AspectJ для решения этой задачи. Данный фремворк позволяет налету модифицировать байт-код классов во время их загрузки в JVM. Для того, чтобы "подцепиться" к нужным нам методам, надо описать точки входа (pointcuts) и аспекты (aspects): 

\begin{lstlisting}
@Pointcut("@annotation(ru.yandex.qatools.allure.annotations.Step)")
public void withStepAnnotation() {
    //pointcut body, should be empty
}

@Pointcut("execution(* *(..))")
public void anyMethod() {
    //pointcut body, should be empty
}

@Before("anyMethod() && withStepAnnotation()")
public void stepStart(JoinPoint joinPoint) {
    ...
}

@AfterThrowing(pointcut = "anyMethod() && withStepAnnotation()", throwing = "e")
public void stepFailed(JoinPoint joinPoint, Throwable e) {
    ...
}

@AfterReturning(pointcut = "anyMethod() && withStepAnnotation()", returning = "result")
public void stepStop(JoinPoint joinPoint, Object result) {
    ...
}
\end{lstlisting}


\startconclusionpage

В ходе работы поставленные задачи были достигнуты. Allure фреймворк стал незаменимой частью отдела тестирования компании Яндекс, и на данный момент активно продолжает развиваться.


\printbibliography

%\startappendices
%\begin{lstlisting}[style=XML, caption=Пример подключения Allure к JUnit тестам с использованием Maven]
<?xml version="1.0" encoding="UTF-8"?>
<project xmlns="http://maven.apache.org/POM/4.0.0"
         xmlns:xsi="http://www.w3.org/2001/XMLSchema-instance"
         xsi:schemaLocation="http://maven.apache.org/POM/4.0.0 http://maven.apache.org/xsd/maven-4.0.0.xsd">
    <modelVersion>4.0.0</modelVersion>
    <parent>
        <groupId>org.sonatype.oss</groupId>
        <artifactId>oss-parent</artifactId>
        <version>7</version>
    </parent>

    <groupId>ru.yandex.qatools.allure</groupId>
    <artifactId>allure-junit-example</artifactId>
    <version>1.0-SNAPSHOT</version>

    <properties>
        <allure.version>1.3.6</allure.version>
        <aspectj.version>1.7.4</aspectj.version>

        <compiler.version>1.7</compiler.version>
        <project.build.sourceEncoding>UTF-8</project.build.sourceEncoding>
    </properties>

    <name>Allure Example Unsing jUnit and WebDriver</name>
    <description>Allure Example Unsing jUnit and WebDriver</description>

    <dependencies>
        <dependency>
            <groupId>ru.yandex.qatools.allure</groupId>
            <artifactId>allure-junit-adaptor</artifactId>
            <version>${allure.version}</version>
        </dependency>
        <dependency>
            <groupId>com.github.detro.ghostdriver</groupId>
            <artifactId>phantomjsdriver</artifactId>
            <version>1.0.4</version>
        </dependency>
    </dependencies>

    <build>
        <plugins>
            <plugin>
                <groupId>org.apache.maven.plugins</groupId>
                <artifactId>maven-archetype-plugin</artifactId>
                <version>2.2</version>
            </plugin>
            <plugin>
                <groupId>org.apache.maven.plugins</groupId>
                <artifactId>maven-compiler-plugin</artifactId>
                <version>3.0</version>
                <configuration>
                    <source>${compiler.version}</source>
                    <target>${compiler.version}</target>
                </configuration>
            </plugin>
            <plugin>
                <groupId>org.apache.maven.plugins</groupId>
                <artifactId>maven-surefire-plugin</artifactId>
                <version>2.14</version>
                <configuration>
                    <testFailureIgnore>false</testFailureIgnore>
                    <argLine>
                        -javaagent:${settings.localRepository}/org/aspectj/aspectjweaver/
                        ${aspectj.version}/aspectjweaver-${aspectj.version}.jar
                    </argLine>
                    <properties>
                        <property>
                            <name>listener</name>
                            <value>ru.yandex.qatools.allure.junit.AllureRunListener</value>
                        </property>
                    </properties>
                </configuration>
                <dependencies>
                    <dependency>
                        <groupId>org.aspectj</groupId>
                        <artifactId>aspectjweaver</artifactId>
                        <version>${aspectj.version}</version>
                    </dependency>
                </dependencies>
            </plugin>
        </plugins>
    </build>

    <reporting>
        <excludeDefaults>true</excludeDefaults>
        <plugins>
            <plugin>
                <groupId>ru.yandex.qatools.allure</groupId>
                <artifactId>allure-maven-plugin</artifactId>
                <version>${allure.version}</version>
            </plugin>
        </plugins>
    </reporting>

</project>
\end{lstlisting}
\end{document}
