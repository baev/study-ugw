\chapter{Проектирование программного продукта} 
\label{chapter3}

В предыдущих главах были подробно описаны концепции положенные в основу фремворка Allure. В этой главе
описаны технические подробности реализации.

\section{Подключение к тестам}

Информацию, которую фремворк собирает о тестах, можно разделить на две группы:

\begin{itemize}
\item информация, которую можно получить, не меняя код тестов, например, имя теста, статус выполнения и время выполнения теста;
\item та информация, которую будет предоставлять тестировщик. Например, сценарий теста, описание теста и требования к тесту.
\end{itemize}

Большинство фремворков xUnit предоставляют интерфейс листенера, позволяющий собирать первый тип информации. Данный подход полностью удовлетворяет требованиям, так как подключение листенеров в большинстве случаев вынесено на уровень конфигурации запуска тестов.

Со вторым типом информации намного интереснее. В качестве примера рассмотрим реализацию для jUnit.

\subsection{Шаги}

В языке программирования Java шаг теста это любой метод, аннотированный аннотацией @Step. Для того, чтобы собрать информацию о пройденных шагах, надо выполнять некоторый код до и после каждого вызова такого метода. Автор воспользовался фреймворком AspectJ для решения этой задачи. Данный фремворк позволяет налету модифицировать байт-код классов во время их загрузки в JVM. Для того, чтобы "подцепиться" к нужным нам методам, надо описать точки входа (pointcuts) и аспекты (aspects): 

\begin{lstlisting}
@Pointcut("@annotation(ru.yandex.qatools.allure.annotations.Step)")
public void withStepAnnotation() {
    //pointcut body, should be empty
}

@Pointcut("execution(* *(..))")
public void anyMethod() {
    //pointcut body, should be empty
}

@Before("anyMethod() && withStepAnnotation()")
public void stepStart(JoinPoint joinPoint) {
    ...
}

@AfterThrowing(pointcut = "anyMethod() && withStepAnnotation()", throwing = "e")
public void stepFailed(JoinPoint joinPoint, Throwable e) {
    ...
}

@AfterReturning(pointcut = "anyMethod() && withStepAnnotation()", returning = "result")
public void stepStop(JoinPoint joinPoint, Object result) {
    ...
}
\end{lstlisting}

