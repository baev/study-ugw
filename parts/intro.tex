\startprefacepage

Тестирование программного обеспечения --- важная часть процесса разработки. Последнее время большинство проектов не могут обходиться без него. Тестирование --- это процесс исследования, испытания программного продукта, имеющий две различные цели:

\begin{itemize}
\item продемонстрировать разработчикам и заказчикам, что программа соответствует требованиям;
\item выявить ситуации, в которых поведение программы является неправильным, нежелательным или не соответствующим спецификации.  
\end{itemize}

Есть несколько уровней тестирования:

\begin{itemize}
\item Модульное тестирование (юнит-тестирование) --- тестируется минимально возможный для тестирования компонент, например, отдельный класс или функция. Часто модульное тестирование осуществляется разработчиками ПО.
\item Интеграционное тестирование --- тестируются интерфейсы между компонентами, подсистемами или системами. При наличии резерва времени на данной стадии тестирование ведётся итерационно, с постепенным подключением последующих подсистем.
\item Системное тестирование --- тестируется интегрированная система на ее соответствие требованиям.
\end{itemize}

По сути, процесс тестирования состоит из двух частей: непосредственно проведение тестов и анализ результатов. Если в случае модульного тестирование анализ результатов занимает незначительное время, то при интеграционном тестировании это весьма нетривиальная задача. И когда количество высокоуровневые тестов в проекте начинает расти, анализ результатов становится большой проблемой. Дело в том, что высокоуровневые тесты сильно отличаются от модульных, и обладают рядом особенностей:

\begin{itemize}
\item они затрагивают гораздо больше функциональности, что затрудняет локализацию проблемы; 
\item такие тесты воздействуют на систему через посредников, например, браузер;
\item таких тестов очень много, и зачастую приходится вводить дополнительную категоризацию. Это могут быть компоненты, области функциональности, критичность.
\end{itemize}

В рамках стандартной модели xUnit анализировать результаты таких тестов достаточно проблематично. Например, в ошибка фреймворк«Can not click on element «Search Button»» тесте на веб-интерфейс может произойти по следующим причинам:

\begin{itemize}
\item сервис не отвечает;
\item на странице нет элемента «Search Button»;
\item элемент «Search Button» есть, но не получается на него кликнуть.
\end{itemize}

А имея дополнительную информацию о ходе выполнения теста, например, лог работы сервиса и скриншот страницы, локализовать проблему гораздо легче. 

Отсюда возникает следующая задача: разработать такую систему, которая позволяет агрегировать дополнительную информацию о ходе выполнения тестов и строить отчет. 

В данной работе будет описан процесс разработки такой системы.

Во второй главе, на основании анализа различных систем построения отчетов автотестов, а также опыта написания тестирования, сформулированы основные принципы для организации системы.

В третьей главе приведена подробная архитектура Allure, позволяющая легко интегрироваться с любыми существующими тестовыми фреймворками и расширять имующийся функционал. Подробно описана интеграция новых фреймворков, и новых систем сборки.

В заключении дано описание текущего состояния разработки и перспективы ее развития.