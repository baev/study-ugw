\startprefacepage

Процесс проектирования и разработки больших программных систем зачастую весьма сложен и тестно связан с процессом контроля его качества. Один из способов контроля качества программного обеспечения это тестирование. Тестирование --- это процесс исследования, испытания программного продукта, имеющий две различные цели:

\begin{itemize}
\item продемонстрировать разработчикам и заказчикам, что программа соответствует требованиям;
\item выявить ситуации, в которых поведение программы является неправильным, нежелательным или не соответствующим спецификации.  
\end{itemize}

Часто процесс тестирования автоматизируют. Это полезно, например, для регрессионного тестирования. Регрессионное тестирование --- собирательное название для всех видов тестирования программного обеспечения, направленных на обнаружение ошибок в уже протестированных участках исходного кода. Написание автотестов в таком случае помогает избежать многочисленных проверок той же функциоанльности в будущем. Однако с процессом развития программной системы количество таких тестов может сильно возрасти. 

Есть несколько уровней тестирования:

\begin{itemize}
\item Модульное тестирование (юнит-тестирование) --- тестируется минимально возможный для тестирования компонент, например, отдельный класс или функция. Часто модульное тестирование осуществляется разработчиками ПО.
\item Интеграционное тестирование --- тестируются интерфейсы между компонентами, подсистемами или системами. При наличии резерва времени на данной стадии тестирование ведётся итерационно, с постепенным подключением последующих подсистем.
\item Системное тестирование --- тестируется интегрированная система на ее соответствие требованиям.
\end{itemize}

Большие программные системы для обеспечения высокого уровня качества зачастую проходят несколько этапов тестирования. А сложность высокоуровневых тестов на такие системы может быть сравнима со сложностью тестируемых систем. Высокоуровневые тесты сильно отличаются от модульных, и обладают рядом особенностей:

\begin{itemize}
\item они затрагивают гораздо больше функциональности, что затрудняет локализацию проблемы; 
\item такие тесты воздействуют на систему через посредников, например, браузер;
\item таких тестов очень много, и зачастую приходится вводить дополнительную категоризацию. Это могут быть компоненты, области функциональности, критичность.
\end{itemize}

Процесс тестирования состоит из двух этапов: непосредственно проведение тестов и анализ результатов. Если речь идет о модульном тестировании программных систем, то анализ результатов занимает незначительное время. Но если говорить о функциональном тестировании, то это не так. Функциоанльное тестирование программных систем --- достаточно сложная задача. И часто существенную часть времени тестировщика занимает именно анализ результатов тестирования. В рамках данной работы разработывается система, позволяющая сократить время анализа результатов тестирования программных систем.

Во второй главе, на основании анализа различных систем построения отчетов автотестов, а также опыта тестирования, сформулированы основные принципы для организации системы.

В третьей главе приведена подробная архитектура Allure, позволяющая легко интегрироваться с любыми существующими тестовыми фреймворками и расширять имующийся функционал. Подробно описана интеграция новых фреймворков, и новых систем сборки.

В заключении дано описание текущего состояния разработки и перспективы ее развития.