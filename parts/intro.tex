\startprefacepage

Бывают модульные тесты, а бывают высокоуровневые. И когда их количество начинает расти, анализ результатов тестов становится проблемой. Дело в том, что высокоуровневые тесты сильно отличаются от модульных, и обладают рядом особенностей:

\begin{itemize}
\item они затрагивают гораздо больше функциональности, что затрудняет локализацию проблемы; 
\item такие тесты воздействуют на систему через посредников, например, браузер;
\item таких тестов очень много, и зачастую приходится вводить дополнительную категоризацию. Это могут быть компоненты, области функциональности, критичность.
\end{itemize}

В рамках стандартной модели xUnit анализировать результаты таких тестов достаточно проблематично. Например, в ошибка «Can not click on element «Search Button»» тесте на web-интерфейс может произойти по следующим причинам:

\begin{itemize}
\item сервис не отвечает;
\item на странице нет элемента «Search Button»;
\item элемент «Search Button» есть, но не получается на него кликнуть.
\end{itemize}

А имея дополнительную информацию о ходе выполнения теста, например, лог работы сервиса и скриншот страницы, локализовать проблему гораздо легче. 

Отсюда возникает следующая задача: разработать такую систему, которая позволяет агрегировать дополнительную информацию о ходе выполнения тестов и строить отчет. 

В данной работе будет описан процесс разработки такой системы.

Во второй главе, на основании анализа различных систем построения отчетов автотестов, а также опыта написания тестирования, сформулированы основные принципы для организации системы.

В третьей главе приведена подробная архитектура Allure, позволяющая легко интегрироваться с любыми существующими тестовыми фремворками и расширять имующийся функционал. Подробно описана интеграция новых фремворков, и новых систем сборки.

В заключении дано описание текущего состояния разработки и перспективы ее развития.