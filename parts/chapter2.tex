\chapter{Исследование}
\label{chapter2}

В данной главе рассматривается теоритические аспекты разработки фремворка. Проводится исследование существующих систем, описываются требования к разрабатываемой системе.

\section{JUnit}

Рассмотрим более подробно тестовый фреймворк JUnit. Данный фреймворк впервые показал, как надо устраивать процесс тестирования (на самом деле, основные идеи были сформированы Кент Беком при разработке SUnit, но именно в лице JUnit эти идеи получили широкое распространение):

\begin{itemize}
\item тесты представляют из себя набор проверок утверждений;
\item тесты могут быть сгруппированы в суиты, для совместного запуска;
\item у каждого теста или суита может быть подготовка (set up) или завершение (tear down);
\item суиты объеденяются в тест ран.
\end{itemize}

Ниже можно посмотреть пример простейщего JUnit теста.

\begin{lstlisting}[caption=Простой JUnit тест.]
public class SampleTest {

    @Test
    public void sampleTest() throws Exception {
        assertThat(4, is(2 + 2));
    }
}
\end{lstlisting}

Для семейства xUnit существует стандартный отчет Surefire. Есть несколько разных видов этого отчета, но суть у всех одна: отображается список всех тестов, у каждого теста есть статус, время выполнения и сообщение об об ошибке (в случае, если тест не прошел). 

Основным недостатком xUnit является атомарность теста, невозможность отобразить тестовый сценарий. Основаный на JUnit тестовый фремворк Thucydides решает эту проблему путем разбиения тестов на шаги.

\section{Thucydides}

Thucydides это фремворк для написание тестов на веб-интерфейс с использованием webDriver, написанный Джоном Смартом (John Ferguson Smart). Джон Смарт --- специалист в BDD, в оптимизации жизненного цикла процесса разработки. Хорошо известеный спикер множества интернациональных конференций, автор множества статей.

В свое время данный фреймворк произвел революцию. Прежде всего, фремворк предлагал структуру для тестов на веб-интерфейс, концепцию разбиения тестов на шаги и возможность сохранять скриншоты каждого шага. Шагом теста являлся любой метод, проаннотированный аннотацией @Step.
Фремворк анализировал структуру данных методов и отображал информацию о них в отчете. Мало того, это помогало сильно сократить код тестов --- шаги выносились в отдельные библиотеки и переиспользовались в множестве проектов.

Отчет, который строил Thucydides для тестов, могли посмотреть другие люди, и понять, что происходит в тесте. В Яндексе это позволило разделить тестировщиков на "автоматизаторов" и ответственных за релиз. Первые писали тесты, а вторые эти тесты запускали, просматривали результаты и, в случае необходимости, дополнительно проводили ручное тестирование продукта. Это позволило сильно увеличить качество тестирования за счет появления дополнительного уровня контроля тестировщиков.

Еще одной важной возможностью Thucydides являлось сохранение скриншотов. В тестах на веб-интерфейс очень важно иметь возможно увидеть, в чем проблема. Однако, не всегда хватает возможности сохранения только скриншотов. Хотелось бы уметь приклеплять к отчету и другие типы данных, например, логи.

Не обошлось в данном фреймворке без недостатков. Прежде всего, предложеная структура слишком ограничивала возможности тестирования. С использованием Thucydides можно писать тесты на веб-интерфейс, а это лишь малая часть функционального тестирования. Для тестов, например, на API данный фреймворк не подходит.

Также важным недостатком является ограничение в используемых технологиях: тесты можно писать только используя тестовый фреймворк JUnit и систему сборки Maven. Хотя, наверное, в любой большой компании тесты пишутся на разных языках программирования, в зависимости от специфики поставленной задачи.  

Кроме того, Thucydides предлагал очень много возможностей, со временем превращаясь в большой проект с множеством проблем. Правильным решением было бы сделать модульный проект, с возможностью подмены некоторых компонент.

\subsection{Итого} 

Thucydides имеет следующие плюсы: 
\begin{itemize}
\item задается единая структура тестов;
\item разбиение тестов на шаги, отображение сценариев тестов;
\item сохранение скриншотов;
\item хороший отчет, с возможностью группировки тестов по требованиям.
\end{itemize}

Но минусов тоже много:
\begin{itemize}
\item слишком много ограничений, заданных структурой тестов;
\item слишком много ограничений на структуру шагов;
\item невозможность использовать другие типы аттачментов;
\item большое количество проблем и ошибок;
\item отчет понятен только тестировщикам.
\end{itemize}

Вывод: отличный фремворк, если речь идет только о простом тестировании веб-интерфейсов с использованием JUnit и Maven. В иных случаях не подходит.

\section{Уточненные требования к работе}

Основываясь на анализе существующих решений, были выработаны уточненные требования к фремворку Allure:

\begin{itemize}
\item умение оперировать как в терминах xUnit, так и в терминах BDD;
\item отображение сценария теста, разбиение теста на шаги;
\item возможность приклеплять к результатам теста произвольные данные;
\item независимость от стека используемых технологий;
\item простой и понятный отчет, который смогут смотреть не только разработчики тестов.
\end{itemize}

В следующей главе рассказывается о процессе разработки фреймворка, удовлетворяющего данным требованиям, о проблемах, с которыми пришлось столкнуться автору данной работы в процессе разработки и описывается фреймворка на момент написания работы.