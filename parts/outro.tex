\startconclusionpage

В ходе работы поставленные задачи были успешно решены.

\begin{itemize}
\item Отчет Allure может группирировать результаты выполнения тестов как в терминах xUnit, так и в терминах BDD. Также есть возможность группировки по тексту ошибок.
\item В отчете есть возможность отображать сценарий теста, разбивать тест на шаги.
\item Еще предоставяется возможность приклеплять к результатам теста произвольные данные.
\item Allure независит от стека используемых технологий, есть возможность использовать его с различными языками программирования, тестовыми фреймворками и системами сборки (интеграции). 
\item Одним из основных преимуществ отчета является прозрачное и легкое подключение к тестам.
\item Получившийся отчет действительно является простым и понятным. Отчет могут смотреть как тестировщики, так и разработчики или менеджеры. Мало того, в отчете может разобраться человек, далекий от программирования.
\end{itemize}

Allure фреймворк стал незаменимой частью отдела тестирования компании Яндекс, и на данный момент активно продолжает развиваться. 

Allure был адаптирован под основные тестовые фреймворки, которые используются для функционального тестирования, за исключением C sharp, и под основные системы выполнения тестов. 

Была разработана уникальная архитектура, позволяющая писать самодокументирующиеся тесты.

В ходе работы автор изучил следующие технологии:

\begin{itemize}
\item Инструментирование кода: cglib, Spring Aspects, aspectJ, OW2 ASM.
\item Написание плагинов: Maven, Jenkins, TeamCity.
\item Java SPI.
\item JUnit (автор работы является контрибутором), TestNG, PyTest.
\item Java шаблонизатор Freemarker.
\item XSLT, XQuery, XSD, JAXB.
\item написание Command Line Interface.
\item AngularJS, Jasmine.
\item Непрерывная интеграция на примере Github + TeamCity.
\end{itemize}

 
фреймворк