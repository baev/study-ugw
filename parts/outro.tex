\startconclusionpage

В ходе работы поставленные задачи были достигнуты. Allure фреймворк стал незаменимой частью отдела тестирования компании Яндекс, и на данный момент активно продолжает развиваться.

Allure был адаптирован под основные тестовые фреймворки, которые используются для функционального тестирования, за исключением C sharp, и под основные системы выполнения тестов.

Была разработана уникальная архитектура, позволяющая писать самодокументирующиеся тесты.

В ходе работы автор изучил следующие технологии:

\begin{itemize}
\item Инструментирование кода: cglib, Spring Aspects, aspectJ, OW2 ASM.
\item Написание плагинов: Maven, Jenkins, TeamCity.
\item Java SPI.
\item JUnit (автор работы является контрибутором), TestNG, PyTest.
\item Java шаблонизатор Freemarker.
\item XSLT, XQuery, XSD, JAXB.
\item написание Command Line Interface.
\item AngularJS, Jasmine.
\item Непрерывная интеграция на примере Github + TeamCity.
\end{itemize}

 
